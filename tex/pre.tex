
\documentclass[xcolor=table]{beamer}
\usetheme{default}
\definecolor{darkscarlet}{rgb}{0.34, 0.01, 0.1}
\usecolortheme{default}
\usecolortheme[named=darkscarlet]{structure}
\setbeamercolor{title}{bg=white, fg=darkscarlet}
\definecolor{cerise}{rgb}{0.87, 0.19, 0.39}
\hypersetup{colorlinks=TRUE,linkcolor=cerise,urlcolor=cerise, citecolor=cerise}

% nummering

\setbeamertemplate{caption}[numbered]

% taal

\usepackage[english, russian]{babel}

% tekens

\usepackage{fontspec}
\usefonttheme{serif}
\setmainfont[BoldFont=brillb.ttf, ItalicFont=brilli.ttf, BoldItalicFont=brillbi.ttf]{brill.ttf}

% tekst doorstrepen, kennelijk heb je daar een apart pakket voor nodig

\usepackage[normalem]{ulem}

% plaatjes

\usepackage{graphicx}

% tabellen

\usepackage{multirow}

% voorbeelden

\usepackage{philex}

% glossen

\usepackage{leipzig}

\newleipzig{hab}{hab}{habitual}
\newleipzig{ad}{ad}{ad}
\newleipzig{el}{el}{elative}
\newleipzig{aor}{aor}{aorist}
\newleipzig{rfl}{rfl}{reflexive}
\newleipzig{emph}{emph}{emphatic particle}
\newleipzig{pf}{pf}{perfect}
\newleipzig{lat}{lat}{lative}
\newleipzig{sup}{sup}{super}
\newleipzig{atr}{atr}{attributivizer}
\newleipzig{add}{add}{additive}
\newleipzig{in}{in}{in}
\newleipzig{rep}{rep}{reportative}
\newleipzig{msd}{msd}{masdar}
\newleipzig{aff}{aff}{affective}
\newleipzig{indef}{indef}{indefinite}
\newleipzig{trans}{trans}{translative}
\newleipzig{an}{an}{animate}
\newleipzig{inan}{inan}{inanimate}
\newleipzig{th}{th}{thematic element}
\newleipzig{ord}{ord}{ordinal numeral}
\newleipzig{temp}{temp}{temporal converb}
\makeglossaries

% bibliography

\usepackage[backend=biber,babel=other,
        bibstyle=biblatex-sp-unified,
        citestyle=sp-authoryear-comp,
        doi=false,
        maxcitenames=3,
        maxbibnames=99]{biblatex}
\addbibresource{bibliography.bib}


% custom footline

\setbeamerfont{footline}{size=\fontsize{20}{20}\selectfont}

\newcommand{\Ffootline}{\footnotesize
\insertsection
\hfill
\href{https://github.com/sverhees/2019_labseminar_botlikh}{github/sverhees/2019\_labseminar\_botlikh}
\hfill
\insertframenumber/\inserttotalframenumber} 

% custom footline deel 2

\setbeamertemplate{footline}{%
\usebeamerfont{structure}
\begin{beamercolorbox}[wd=\paperwidth,ht=2.25ex,dp=1ex]{title in head/foot}%
\Tiny\hspace*{4mm} \Ffootline \hspace{4mm}
\end{beamercolorbox}}

% navigatiesymbolen uitzetten

\beamertemplatenavigationsymbolsempty
 
 
% titelpagina 

\title{Fieldtrip to Botlikh}
\author{Samira Verhees \& Chiara Naccarato}
\date{Convergence Lab Seminar - 01.10.2019}


% links

\usepackage{hyperref}

%\newcommand\pro{\item[$+$]}
%\newcommand\con{\item[$-$]}